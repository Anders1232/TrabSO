\documentclass[conference]{IEEEtran}
\providecommand{\e}[1]{\ensuremath{\times 10^{#1}}}
\usepackage{times}
\usepackage[brazil,english]{babel}
\usepackage[utf8]{inputenc}
\usepackage[T1]{fontenc}
\usepackage{indentfirst}
\usepackage{amsmath,amssymb,amsthm}
\usepackage{graphicx,url,subfig}
\usepackage[table,xcdraw]{xcolor} % pacote extra para colocar cores nas celulas da tabel
\usepackage{tikz}
\usepackage{pgfplots}     

\newcommand\tab[1][1cm]{\hspace*{#1}}

\title{GNU Debian}

\author{
\IEEEauthorblockN{Francisco Anderson Bezerra Rodrigues, Marcelo Bulhões Fonseca, Vitor Silva De Deus} \\
\IEEEauthorblockA{Departamento de Ciência da Computação, \\ Universidade de Brasília\\
%\IEEEauthorblockA{Email: anders1232@aluno.unb.br}
}
}
\begin{document} 

\maketitle
\selectlanguage{brazil}  

%\begin{abstract}
%Aqui vai o resumo do artigo! 

%\end{abstract}


\section{Introdução}\label{sec:intro}
Debian é um projeto e um sistema operacional iniciado em 16 Agosto de 1993 por Ian Murdock\cite{DebianHistory}. Cada versão possui o nome de um personagem de Toy Story. É uma distribuição utilizada por várias organizações pelo mundo\cite{DebianUsers} graças à sua establidade, que o torna ideal para servidores.

\section{Propósito do Debian}\label{sec:prop}
Debian é um sistema operacional com o objetivo de ser um sistema operacional completamente livre. No qual qualquer um pode baixar, modificar e compartilhar. De acordo com o projeto GNU~\cite{FreeSoftware}, um software para ser considerado livre, deve permitir que seus usuários possuam todas as seguintes liberdades fundamentais:
\begin{itemize}
	\item Liberdade para rodar o software da forma que o usuário quiser, e para qualquer propósito.
	\item Liberdade para estudar como o programa funciona, e poder editá-lo para que o programa compute da forma que o usuário quiser.
	\item Liberdade para redistribuir cópias.
	\item Liberdade para distribuir suas modificações para outras pessoas.
\end{itemize}
Para que essas liberdades sejam possíveis é necessário que o código fonte do software seja disponibilizado.

O Advanced Packaging Tool(APT) foi criado no projeto Debian e atualmente é usando em várias distribuições. Muitas delas baseadas no próprio Debian, como o Ubuntu e o Mint(que também possui uma versão baseada no Ubuntu).

\section{Requisitos de Aplicações}\label{sec:req}

\section{Linux}\label{sec:linux}

\subsection{Arquitetura}\label{sec:LinuxArq}

\subsection{Gerência de Memória}\label{sec:LinuxMem}

\subsection{gerência de E/S}\label{sec:LinuxES}

\subsection{gerência de Processos}\label{sec:LinuxPROC}

Os primeiros escalonadores do linux eram bastante simples, sem foco em arquiteturas massivas ou com muitos processadores multithread. O escalonador do linux 1.2 usava uma fila circular com Round-Robin. No linux 2.2 foi inserido outro escalonador com políticas de escalonamento para: tarefas de tempo real, tarefas não preemptivas e tarefas de tempo não real, ou seja, classes de escalonamento. Além disso também foi incluído suporte para SMP(symmetric multiprocessing)\cite{LinuxSchedulerIBM}.\\

 No kernel 2.4 foi  inserido outro escalonador relativamente simples com complexidade O(n). Ele era relativamente ineficiente,limitava escalabilidade e era lento para processos de tempo real. Também se mostrou ruim para arquiteturas de multiprocessador. No kernel 2.6,introduziu-se o chamado de escalonador O(1), que tem esse nome por ter complexidade O(1). Esse possuía suporte a SMP e cada fila de execução possuía dois arrays de prioridade.\\

 O array de tarefas ativas e o de expiradas. No caso,cada array contém uma fila de processos por nível de prioridade. Essas filas contém listas de processos prontos em cada nível de prioridade. Os  arrays de prioridade também contém mapas de bits para descobrir eficientemente a maior prioridade entre as tarefas prontas no sistema sem precisar percorrer as filas. Achar a próxima tarefa a ser executada é simplesmente selecionar o próximo elemento nas listas.\\

 Dada uma prioridade,as tarefas são escalonadas com Round Robin. Mantêm-se dois arrays de prioridade para cada processador, um para tarefas ativas e outro para expiradas. O de expiradas contém todas as tarefas da lista de execução associada que já exauriram o temporizador. Quando uma tarefa esgota o seu temporizador um novo tempo é atribuído antes dessa tarefa ser movida para o array de expirados. Recalcular todas as temporizações é tão simples quanto trocar o array de expirados pelo de não expirados porque os arrays são acessados apenas por ponteiro. Trocar os arrays é tão rápido quanto trocar dois ponteiros\cite{LinuxSchedulerMIT}.\\

 O escalonador O(1) era muito mais escalável e incorporou interatividade com várias métricas heurísticas para determinar se as tarefas eram IO-bound ou process-Bound. Apesar das vantagens o escalonador O(1) tornou-se pesado no kernel. A grande massa de código necessária para calcular heurísticas tornou-se difícil de manejar.\\

 A nova mudança veio por meio do algoritmo RSDL(Rotating Staircase Deadline Scheduler), proposto por Con Kolivas e foi incorporado a versão 2.6.21 do kernel linux. Depois, Ingo Molnar,o criador do escalonador O(1) desenvolveu um algoritmo CFS baseado nas mesmas ideias do trabalho de Kolivas. A principal ideia por trás do CSF é manter a justiça no provimento do processador aos processos. Para manter esse balanceamento, a quantidade de tempo designada a uma tarefa no CFS é chamada de virtual-runtime. Quanto menor for o "virtual-runtime" de uma tarefa maior a necessidade que essa tarefa tem de ser executada.\\
 
O CFS também possui uma politica de justiça processos bloqueados(sleeper fainess) para processos que estão parados esperando por I/O tenham algum tempo para execução quando precisarem, por exemplo.Entretanto, ao invés de manter as tarefas em uma fila de espera, o CFS mantem uma árvore "red-black" ordenada por tempo.\\

Uma árvore "red-black" é uma árvore auto-balanceável, as operações na árvore ocorrem com complexidade O(log n), onde n é o número de nós da árvore. Nessa estrutura, tarefas com maior necessidade do processador(menor virtual-runtime) são armazenadas no lado esquerdo da arvore, e tarefas com menor necessidade de serem executadas do lado direito \cite{LinuxSchedulerIBM}. O escalonamento ocorre como a seguir:
\begin{enumerate}
	\item O nó mais a esquerda da árvore é escolhido e enviado à execução.
	\item Se o processo conclui sua execução ele é removido do sistema e da árvore
	\item Se o processo alcança seu tempo máximo de execução(quantum) ou é bloqueado(voluntariamente ou por interrupção), ele é reinserido na árvore baseado em seu novo "virtual-runtime".
	\item O novo nó mais a esquerda será escolhido repetindo a iteração\cite{LinuxSchedulerWiki}
\end{enumerate}



Outra alternativa moderna ao escalonador O(1) é o BFS(Brain fuck scheduler). Também desenvolvido por Con Kolivas, esse escalonador evita o uso de configurações e heurísticas. EM alguns benchmarks apresenta desempenho levemente ou moderadamente maior que o CFS mas não chegou a tomar o seu lugar. Está presente em distribuições como Sabayon Linux\cite{LinuxSchedulerBFS}.
\cite{LinuxSchedulerWiki}

\subsection{Funcionamento de interrupções}\label{sec:LinuxInt}

\subsection{Suporte a Threads}\label{sec:LinuxThreads}
O linux dá suporte a threads tratando-as como processos e escalonando-os juntamente. Um processo pode ser visto como uma thread mas
processos podem contem múltiplas threads que compartilham seus recursos\cite{LinuxSchedulerIBM}.
\subsection{Segurança}\label{sec:LinuxSec}

\section{FreeBSD}\label{sec:BSD}

\subsection{Arquitetura}\label{sec:BSDArq}

\subsection{Gerência de Memória}\label{sec:BSDMem}
O FreeBSD usa uma política de troca de páginas ao estilo Last Recent Used(LRU)\cite{FreeBsdArc}

\subsection{gerência de E/S}\label{sec:BSDES}

\subsection{gerência de Processos}\label{sec:BSDPROC}

\subsection{Funcionamento de interrupções}\label{sec:BSDInt}

\subsection{Suporte a Threads}\label{sec:BSDThreads}

\subsection{Segurança}\label{sec:BSDSec}

\section{Hurd}\label{sec:Hurd}

\subsection{Arquitetura}\label{sec:HurdArq}
O kernel GNU/Hurd possui a arquitetura cliente/servidor e possui o microkernel GNU/Mach que contém instruções de modo kernel. No Hurd, pode-se atrelar à arquivos, máquinas de tradução. Que podem prover diversas funcionalidades, como por exemplo implementar o protocolo ftp de forma transparente, prover gerenciamento de histórico de arquivos entre outras coisas muito interessantes.\cite{HurdPaper}

\subsection{Gerência de Memória}\label{sec:HurdMem}

\subsection{gerência de Processos}\label{sec:HURDPROC}

\subsection{gerência de E/S}\label{sec:HurdES}
No GNU/Hurd as operações de entrada e saída são feitas com o uso de portas de comunicação que implementam pelo menos o protocolo de arquivo ou o protocolo de socket. As operações são realizadas a partir de chamadas de procedimeto remoto(RPC's) nessas portas de comunicação\cite{HurdIO}.

\subsection{Funcionamento de interrupções}\label{sec:HurdInt}

\subsection{Suporte a Threads}\label{sec:HurdThreads}
O GNU/Hurd faz uso agressivo de Threads como o objetivo de extrair o máximo do hardware.\cite{Hurd} Todas as bibliotecas do GNU/Hurd são thread-safe, ou seja, chamadas cocorrentes aos serviços da biblioteca não fazem ela se comportar de forma inesperada.

\subsection{Segurança}\label{sec:HurdSec}
Um dos objetivos do Hurd é minimizar a quantidade de código que é executada no modo kernel através do uso de microkernel GNU Mach. O que aumenta a confiabilidade do sistema como um todo.

\bibliographystyle{unsrt}
\bibliography{arquivo.bib}
%\end{multicols*} %fim do texto em 2 duas colunas

\end{document}
