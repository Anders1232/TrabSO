\documentclass[conference]{IEEEtran}
\providecommand{\e}[1]{\ensuremath{\times 10^{#1}}}
\usepackage{times}
\usepackage[brazil,english]{babel}
\usepackage[utf8]{inputenc}
\usepackage[T1]{fontenc}
\usepackage{indentfirst}
\usepackage{amsmath,amssymb,amsthm}
\usepackage{graphicx,url,subfig}
\usepackage[table,xcdraw]{xcolor} % pacote extra para colocar cores nas celulas da tabel
\usepackage{tikz}
\usepackage{pgfplots}     



\title{Laboratório ICMP - Relatório}

\author{
\IEEEauthorblockN{Francisco Anderson Bezerra Rodrigues} \\
\IEEEauthorblockA{Departamento de Ciência da Computação, \\ Universidade de Brasília\\
\IEEEauthorblockA{Email: anders1232@aluno.unb.br}}
}
\begin{document} 

\maketitle
\selectlanguage{brazil}  

\begin{abstract}
Aqui vai o resumo do artigo! 

\end{abstract}


\section{Introdução}\label{sec:intro}
Aqui inicia o artigo artigo aushśaidvbklzxnxkversnr. Exemplo de referência no texto~\cite{Vmware}, 

\bibliographystyle{unsrt}
\bibliography{arquivo.bib}
%\end{multicols*} %fim do texto em 2 duas colunas

\end{document}
