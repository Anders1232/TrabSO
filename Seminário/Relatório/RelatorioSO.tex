\documentclass[conference]{IEEEtran}
\providecommand{\e}[1]{\ensuremath{\times 10^{#1}}}
\usepackage{times}
\usepackage[brazil,english]{babel}
\usepackage[utf8]{inputenc}
\usepackage[T1]{fontenc}
\usepackage{indentfirst}
\usepackage{amsmath,amssymb,amsthm}
\usepackage{graphicx,url,subfig}
\usepackage[table,xcdraw]{xcolor} % pacote extra para colocar cores nas celulas da tabel
\usepackage{tikz}
\usepackage{pgfplots}     



\title{GNU Debian}

\author{
\IEEEauthorblockN{Francisco Anderson Bezerra Rodrigues, Marcelo Bulhões Fonseca, Vitor Silva De Deus} \\
\IEEEauthorblockA{Departamento de Ciência da Computação, \\ Universidade de Brasília\\
%\IEEEauthorblockA{Email: anders1232@aluno.unb.br}
}
}
\begin{document} 

\maketitle
\selectlanguage{brazil}  

%\begin{abstract}
%Aqui vai o resumo do artigo! 

%\end{abstract}


\section{Introdução}\label{sec:intro}
Debian é um projeto e um sistema operacional iniciado em 16 Agosto de 1993 por Ian Murdock\cite{DebianHistory}. Cada versão possui o nome de um personagem de Toy Story. É uma distribuição utilizada por várias organizações pelo mundo\cite{DebianUsers} graças à sua establidade, que o torna ideal para servidores.

\section{Propósito do Debian}\label{sec:prop}
Debian é um sistema operacional com o objetivo de ser um sistema operacional completamente livre. No qual qualquer um pode baixar, modificar e compartilhar. De acordo com o projeto GNU~\cite{FreeSoftware}, um software para ser considerado livre, deve permitir que seus usuários possuam todas as seguintes liberdades fundamentais:
\begin{itemize}
	\item Liberdade para rodar o software da forma que o usuário quiser, e para qualquer propósito.
	\item Liberdade para estudar como o programa funciona, e poder editá-lo para que o programa compute da forma que o usuário quiser.
	\item Liberdade para redistribuir cópias.
	\item Liberdade para distribuir suas modificações para outras pessoas.
\end{itemize}
Para que essas liberdades sejam possíveis é necessário que o código fonte do software seja disponibilizado.

\section{Requisitos de Aplicações}\label{sec:req}

\section{Linux}\label{sec:linux}

\subsection{Arquitetura}\label{sec:LinuxArq}

\subsection{Gerência de Memória}\label{sec:LinuxMem}

\subsection{gerência de E/S}\label{sec:LinuxES}

\subsection{Funcionamento de interrupções}\label{sec:LinuxInt}

\subsection{Suporte a Threads}\label{sec:TinuxThreads}

\subsection{Segurança}\label{sec:LinuxSec}

\section{FreeBSD}\label{sec:BSD}

\subsection{Arquitetura}\label{sec:BSDArq}

\subsection{Gerência de Memória}\label{sec:BSDMem}

\subsection{gerência de E/S}\label{sec:BSDES}

\subsection{Funcionamento de interrupções}\label{sec:BSDInt}

\subsection{Suporte a Threads}\label{sec:BSDThreads}

\subsection{Segurança}\label{sec:BSDSec}

\section{Hurd}\label{sec:Hurd}

\subsection{Arquitetura}\label{sec:HurdArq}

\subsection{Gerência de Memória}\label{sec:HurdMem}

\subsection{gerência de E/S}\label{sec:HurdES}

\subsection{Funcionamento de interrupções}\label{sec:HurdInt}

\subsection{Suporte a Threads}\label{sec:HurdThreads}

\subsection{Segurança}\label{sec:HurdSec}


\bibliographystyle{unsrt}
\bibliography{arquivo.bib}
%\end{multicols*} %fim do texto em 2 duas colunas

\end{document}
